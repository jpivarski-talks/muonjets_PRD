\documentclass[prd,twocolumn]{revtex4}
\usepackage{graphicx}
\usepackage{amssymb}

\begin{document}
\title{Search for Resonant Production of Muonic Lepton Jets}
\author{CMS Collaboration}
\noaffiliation

\date{\today}

\begin{abstract}
A signature-based search for groups of collimated muons (muonic lepton
jets) is performed using 35~pb$^{-1}$ of data collected by the CMS
experiment at the LHC, at a center-of-mass energy of 7~TeV.  The
analysis inclusively searches for production of new low-mass states
decaying into pairs of muons, and is designed to achieve high
sensitivity to a broad range of models predicting muonic lepton jet
signatures.  With no excess observed in the data over the background
expectation, upper limits on the production cross-section times
branching ratio times acceptance are derived for several event
topologies and range from 0.1 to 0.5~pb.  In addition, the results are
interpreted for several benchmark models in the context of SUSY with a
low-mass dark sector, yielding limits on new physics exceeding the
Tevatron reach.
\end{abstract}

\maketitle

\section{Introduction}

hey \citep{Pamela-positron}

\subsection{Motivation}

\subsection{Model-independent strategy}

\subsection{Benchmark dark matter models}

\section{Analysis selection and efficiency}

\subsection{Dataset and trigger}

\subsection{Offline selections and analysis}

\subsection{Systematic uncertainty in efficiency}

\section{Background and signal shapes}

\subsection{Features of the low-mass dimuon spectrum}

\subsection{Shape of backgrounds in $R^1_2$}

\subsection{Shape of backgrounds in $R^1_4$}

\subsection{Shape of backgrounds in $R^2_{22}$}

\subsection{Shape of signals}

\section{Statistical interpretation of the results}

\section{Results}

\subsection{Model-independent limits}

\subsection{Limits on benchmark models}

\section{Conclusions}

\appendix

\section{Event displays}

\bibliographystyle{apsrev}
\bibliography{prd}

\end{document}


%% Paraphotons
%% H. Georgi, S. Glashow and P. Ginsparg, Nature 306 (1983) 765
%% B. Holdom, Phys. Lett. B166 (1986) 196
%% S. Coleman and S. Glashow, Phys. Lett. B405 (1997) 249
%% S. Glashow, Phys. Lett. B430 (1998) 54
